\usepackage{amsmath,amssymb,amsfonts,amsthm}
\usepackage{mathtools,bbm}
\usepackage{IEEEtrantools}
\usepackage[]{algorithm2e}
\usepackage{tikz,pgfplots,tkz-graph}
\usepackage{listings,hyperref,float}
\usepackage{epsfig,epstopdf,graphicx}
\usepackage{enumerate}

\pgfplotsset{compat=newest}
\usetikzlibrary{positioning,matrix}
\interdisplaylinepenalty=2500

% Page settings
\setlength{\oddsidemargin}{0 in}
\setlength{\evensidemargin}{0 in}
\setlength{\topmargin}{-0.6 in}
\setlength{\textwidth}{6.5 in}
\setlength{\textheight}{8.5 in}
\setlength{\headsep}{0.75 in}
\setlength{\parindent}{0 in}
\setlength{\parskip}{0.1 in}

% Markov symbol X -o- Y
\newcommand{\markov}{\mathrel{\multimap}\joinrel\mathrel{-}\joinrel\mathrel{\mkern-6mu}\joinrel\mathrel{-}}
% Independent & not independent symbol X _||_ Y
\newcommand{\indep}{\mathrel{\bot}\joinrel\mathrel{\mkern-5mu}\joinrel\mathrel{\bot}}
\newcommand{\dep}{\centernot\indep}
% i.i.d.
\newcommand{\iid}{\stackrel{\mathrm{i.i.d.}}{\sim}}
% Integration d
\newcommand{\dd}{\mathop{}\!\mathrm{d}}

% Single braces
\newcommand{\agbrs}[1]{\left\langle #1 \right\rangle}	% angle braces,  	< x >
\newcommand{\rdbrs}[1]{\left( #1 \right)}				% round braces,  	( x )
\newcommand{\sqbrs}[1]{\left\lbrack #1 \right\rbrack}	% square braces, 	[ x ]
\newcommand{\clbrs}[1]{\left\lbrace #1 \right\rbrace}	% curly braces, 	{ x }

% Delimited braces
\newcommand{\agbrsv}[2]{\left\lange #1 \,\middle|\, #2 \right\rangle}	% angle braces with delimiter |,  	< x | y >, inner product
\newcommand{\rdbrsv}[2]{\left( #1 \,\middle|\, #2 \right)}				% round braces with delimiter |,  	( x | y ), conditional probability
\newcommand{\sqbrsv}[2]{\left\lbrack #1 \,\middle|\, #2 \right\rbrack}	% square braces with delimiter |,  	[ x | y ], conditional expectation
\newcommand{\clbrsv}[2]{\left\lbrace #1 \,\middle|\, #2 \right\rbrace}	% curly braces with delimiter |,  	{ x | y }, set

% Short for underline and overline
\newcommand{\udl}[1]{\underline{#1}}
\newcommand{\ovl}[1]{\overline{#1}}

% Derivative and partial derivation
\newcommand{\fracd}[2]{ \frac{\mathrm{d} #1}{\mathrm{d} #2} }
\newcommand{\fracp}[2]{ \frac{\partial #1}{\partial #2} }

% Absolute value, norm, ceil, floor, and evaluation
\newcommand{\abs}[1]{\left| #1 \right|}		% Absolute value, 	| x |
\newcommand{\norm}[1]{\left\| #1 \right\|}	% Norm, 			|| x ||
\DeclarePairedDelimiter\ceil{\lceil}{\rceil}
\DeclarePairedDelimiter\floor{\lfloor}{\rfloor}
\newcommand{\eval}[1]{\left. #1 \right|}	% Evaluation, 		f(x) |_{x=x0}

% Definition equal sign
\newcommand{\defeq}{\vcentcolon=}   % :=
\newcommand{\eqdef}{=\vcentcolon}   % =:
\newcommand{\texteq}[1]{\stackrel{#1}{=\joinrel=\joinrel=}}   % use for change of variable

% Inverse hyperbolic functions
\DeclareMathOperator\arcsinh{arcsinh}
\DeclareMathOperator\arccosh{arccosh}
\DeclareMathOperator\arctanh{arctanh}
\DeclareMathOperator\sech{sech}

% argmax and argmin
\newcommand\limit[1]{\underset{#1}{\lim}\,}
\newcommand\argmax[1]{\mathrm{arg}\,\underset{#1}{\max}\,}
\newcommand\argmin[1]{\mathrm{arg}\,\underset{#1}{\min}\,}

% Matrices
\newcommand*\pmtx[1]{\begin{pmatrix}#1\end{pmatrix}}			% Matrix with round braces
\newcommand*\bmtx[1]{\begin{bmatrix}#1\end{bmatrix}}			% Matrix with square braces
\newcommand*\vmtx[1]{\begin{vmatrix}#1\end{vmatrix}}			% Determinant of matrix
\newcommand*\spmtx[1]{\rdbrs{\begin{smallmatrix}#1\end{smallmatrix}}}	% Small matrix with round braces
\newcommand*\sbmtx[1]{\sqbrs{\begin{smallmatrix}#1\end{smallmatrix}}}	% Small matrix with square braces
\newcommand*\svmtx[1]{\abs{\begin{smallmatrix}#1\end{smallmatrix}}}		% Determinant of small matrix
\newcommand{\sizecorr}[1]{\makebox[0cm]{\phantom{$\displaystyle #1$}}}	% Get size of an expression

% Theorem environments
\newtheorem{theorem}{Theorem}
\newtheorem{corollary}[theorem]{Corollary}
\newtheorem{lemma}[theorem]{Lemma}
\newtheorem{observation}[theorem]{Observation}
\newtheorem{proposition}[theorem]{Proposition}
\newtheorem{definition}[theorem]{Definition}
\newtheorem{claim}[theorem]{Claim}
\newtheorem{fact}[theorem]{Fact}
\newtheorem{assumption}[theorem]{Assumption}

\newtheorem*{theorem*}{Theorem}
\newtheorem*{corollary*}{Corollary}
\newtheorem*{lemma*}{Lemma}
\newtheorem*{proposition*}{Proposition}
\newtheorem*{definition*}{Definition}
\newtheorem*{example*}{Example}
\newtheorem*{remark*}{Remark}
\newtheorem*{problem*}{Problem}

% Solution environment
\newenvironment{solution}{\begin{proof}[Solution]}{\end{proof}}

% Short for boldsymbol
\newcommand{\bs}[1]{\boldsymbol{#1}}

% Other frequent used operators
\def\Var{\mathsf{Var}}
\def\Cov{\mathsf{Cov}}
\def\coeff{\mathsf{coeff}}
\def\Tr{\mathsf{Tr}}
\def\rank{\mathsf{rank}}
\def\diag{\mathsf{diag}}
\def\mse{\mathsf{mse}}
\def\mmse{\mathsf{mmse}}

\def\ee{\mathbbm{e}}
\def\ii{\mathbbm{i}}
\def\EE{\mathsf{E}}
\def\FF{\mathsf{F}}
\def\SS{\mathsf{S}}
\def\UU{\mathsf{U}}
\def\xx{\mathsf{x}}
\def\pprime{{\prime\prime}}

\newcommand{\KL}[2]{D\left( #1 \,\middle|\middle|\, #2 \right)}%                   ( | )

\makeatletter
\newcommand\ztag[1]{%
\def\@currentlabel{#1}%
\gdef\tmp{%
\addtocounter{equation}{-1}%
\def\theequation{#1}}%
\aftergroup\aftergroup\aftergroup\aftergroup\aftergroup\aftergroup
\aftergroup\aftergroup\aftergroup\aftergroup\aftergroup\aftergroup
\aftergroup\aftergroup\aftergroup\aftergroup\aftergroup\aftergroup
\aftergroup\aftergroup\aftergroup\aftergroup\aftergroup\aftergroup
\aftergroup\aftergroup\aftergroup\aftergroup\aftergroup\aftergroup
\aftergroup
\tmp\IEEEyesnumber}
